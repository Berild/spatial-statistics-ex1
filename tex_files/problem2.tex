\paragraph{a)}
text text text

%%%%%%%%%%%%%%%%%%%%%%%%%%%%%%%%%%%%%%%%%%%%%%%%%%%%%%%%%%%%%%%%%%%%%%
\paragraph{b)}
Consider the continuous spatial variables $\{r(\vect{x}); \vect{x} \in \mathcal{D} \subset \mathbb{R}^2\}$ s.t.
\begin{equation}
\begin{array}{rcl}
     E[r(\vect{x})] & = & \vect{g}(\vect{x})^T\vect{\beta}_r \\
     \Var\{r(\vect{x})\} & = & \sigma_r^2 \\
     \Corr\{r(\vect{x}),r(\vect{x}^{'})\} & = & \rho_r(\vect{\tau}/\xi)
\end{array}
\end{equation}
with $\vect{g}(\vect{x}) = (g_1(\vect{x}), ... g_{n_g}(\vect{x}))$ a vector of known spatial variables and $\beta_r = (\beta_1, ..., \beta_{n_g})$ an unknown parameter. $\sigma_r^2 = 2500$ and the spatial correlation function have $\xi = 100$ with $\tau = |\vect{x} - \vect{x^{'}}|$. The prediction is 
\begin{equation}
    \hat{r}_0 = \vect{\alpha}^T\vect{r}_{n_g},
\end{equation}
with $\vect{alpha} = (\alpha_1, ..., \alpha_{n_g})$ being the weights. 
The predictor error needs to be unbiased, which mean that the expected value of the error must be zero, giving
\begin{equation*}
\begin{array}{rcl}
    \E[\hat{r}_0 - r_0] & = & \vect{\alpha}^T \E[r^g]\\
      & \Downarrow & \\
      \vect{\alpha}^T \matr{G}_g \beta_r^+ & = & \vect{g}(\vect{x}_0) \\
      & \Downarrow & \\
      \matr{G}_g^T \vect{\alpha} & = & \vect{g}(\vect{x}_0)
\end{array}
\end{equation*}
with the ($n_g x n_g$)-matrix $\matr{G}_g = (\vect{g}(\vect{x}_1,...,\vect{g}(\vect{x}_{n_g})^T$.
The optimization problem comes from minimizing the prediction error variance. It is derived from
\begin{equation*}
    \begin{array}{rcl}
        \vect{\hat{\alpha}} & = & \arg\min\limits_\alpha \Var\{\hat{r}_0 - r_0\} \\
         & = & \arg\min\limits_\alpha\{\sigma_r^2 - 2\vect{\alpha}^T\sigma_r^2\vect{\rho}_0 + \vect{\alpha}^T\sigma_r^2\matr{\Sigma}_g^\rho \vect{\alpha}\} \\
          & \textrm{constrained by} & \matr{G}_g^T \vect{\alpha} = \vect{g}(\vect{x}_0)
    \end{array}
\end{equation*}
The analytical solution using optimization with lagrange multipliers is
\begin{equation*}
    \vect{\hat{\alpha}} = [\matr{\Sigma}_g^\rho]^{-1}\left[\vect{\rho}_0 - \matr{G}_g^T [\matr{G}_g[\matr{\Sigma}_g^\rho]^{-1}\matr{G}_g^T]^{-1}[\matr{G}_g[\matr{\Sigma}_g^\rho]^{-1}\vect{\rho}_0 - \vect{g}(\vect{x}_0)]\right]
\end{equation*}
which yields the universal kriging predictor
\begin{equation}
    \hat{r}_0 = \vect{\hat{\alpha}}^T \vect{r}_{n_g}
\end{equation}
%%%%%%%%%%%%%%%%%%%%%%%%%%%%%%%%%%%%%%%%%%%%%%%%%%%%%%%%%%%%%%%%%%%%%%
\paragraph{c)}
Let the reference variable $\vect{x} \in \mathcal{D} \subset \mathbb{R}^2$ be $\vect{x} = (x_v, x_h)$ with $n_g = 6$, and $\vect{g}(\vect{x})$ is $x_v^kx_h^l$ for $(k,l) \in \{(0,0),(1,0),(0,1),(1,1),(2,0),(0,2)\}$. Which means that the $n_g$-vector $\vect{g}(\vect{x}) = (1,x_v,x_h,x_vx_h,x_v^2,x_h^2)^T$. Looking at the expected value for $r(\vect{x})$ we get.
\begin{equation*}
    \E[r(\vect{x})] = \vect{g}(\vect{x})^T \vect{\beta}_r = \beta_r^1 + x_v \beta_r^2 + x_h \beta_r^3 + x_v x_h \beta_r^4 + x_v^2 \beta_r^5 + x_h^2 \beta_r^6
\end{equation*}

%%%%%%%%%%%%%%%%%%%%%%%%%%%%%%%%%%%%%%%%%%%%%%%%%%%%%%%%%%%%%%%%%%%%%%
\paragraph{d)}
text text text

%%%%%%%%%%%%%%%%%%%%%%%%%%%%%%%%%%%%%%%%%%%%%%%%%%%%%%%%%%%%%%%%%%%%%%
\paragraph{e)}
text text text

%%%%%%%%%%%%%%%%%%%%%%%%%%%%%%%%%%%%%%%%%%%%%%%%%%%%%%%%%%%%%%%%%%%%%%
\paragraph{f)}
text text text
+